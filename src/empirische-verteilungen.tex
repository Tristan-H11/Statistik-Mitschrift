\chapter{Empirische Verteilungen}
Es handelt sich hier überwiegend um die Bearbeitung von größeren gesammelten Datenmengen, wie man sie ordnet, wie man sie darstellt und wie man sie in Tabellen oder gar Kennzahlen zusammen fassen kann.
Ganz allgemein entstehen diese Daten durch Stichprobenerhebung oder Befragung. Wie die Erhebung oder Befragung organisier wird, ist Sache von Marktforschungsabteilungen.
\clm{Zusammenhang Datenerhebung und Datenqualität}{}{Man muss sich darüber im Klaren sein, dass die Daten niemals besser sein können, als die Erhebungsmethode, aus welcher sie hervorgingen.}
Daraus ergibt sich, dass Marktforscher im Idealfall gute Statistikkenntnisse aufweisen. Diese Vorlesung arbeitet jedoch nach dem Leitsatz: \textit{Statistik beginnt erst, wenn die Daten vorliegen!}

\section{Allgemeines}
\subsection{Entstehung der Daten}
Man muss sich in diesem Rahmen
\begin{enumerate}
    \item zuerst darüber im Klaren sein, welche Frage eigentlich beantwortet werden muss. (\textbf{Forschungshypothese})
    \item dan Gedanken darüber machen, wie diese Frage beantwortet werden kann. Die Entstehung der \textbf{Untersuchungsvariablen}
\end{enumerate}
Um über die Qualität der Operationalisierung bzw der Messung urteilen zu können, müssen Gütekriterien her:
\begin{enumerate}
    \item \textbf{Objektivität:} Merkmalsauswahl bzw deren Auswerten und Interpretation sollte unabhängig vom jeweiligen Forscher erfolgen
    \item \textbf{Zuverlässigkeit:} Wird die Messung wiederholt, sollten ähnliche Ergebnisse rauskommen
    \item \textbf{Gültigkeit:} Messfehler sollten so klein wie möglich sein
\end{enumerate}

\subsection{Merkmale und ihre Eigenschaften}
Die empirische Statistik untersucht die Verteilung von Merkmalen oder Variablen, die nicht vorher durch Theorien bekannt sein können.
Die Ergebnisse sind nicht vorhersehbar, da das gemessene unter den Merkmalsträgern variiert.
\newline
Merkmale können als \textbf{Untersuchungsgegenstand} definiert werden. Es ist eine Eigenschaft, die an einem Untersuchungssubjekt (\textbf{Merkmalsträger}) beobachtet wird.
Die Merkmale und ihre \textbf{Merkmalsausprägungen} werden in der Planung der Untersuchung festgelegt.
\ex{Merkmalsträger und -ausprägung}{Bei einer Umfrage sind die befragten Personen die Merkmalsträger. Die abgefragten Gegenstände (Haarfarbe, Kinderzahl, Alter, Geschlecht) sind die Merkmale. Die Ausprägungen sind entsprechend (blond, schwarz, rot,..) und (0,1,2,..) usw.}
Wie man sieht, können sehr verschiedene Eigenschaften gemessen werden. Eine Untersuchung kann nur eine begrenzte Anzahl von Merkmalen aufnehmen und kann demnach nur als vereinfachtes Abbild der Realität fungieren.
\newline
Bei dem obigen Beispiel sind die Merkmale bewusst so gewählt worden. Es zeigt, dass die Ausprägungen sehr verschiedener Art sein können.
Diese Unterscheidung wird unter dem Begriff \textbf{Messniveau} zusammengefasst:
\dfn{Messniveaus / Skalierungen}{
Es wird in unterschiedliche Skalierungen unterteilt:
\begin{itemize}
    \item \textbf{qualitativ} (nicht-metrische Skalierung)
    \item \textbf{quantitativ} (metrische Skalierung)
\end{itemize}
Die nicht-metrische Skalierung wird wie folgt unterschieden:
\begin{itemize}
    \item \textbf{nominal}: Die Ausprägungen werden lediglich kategorisiert, ohne den Kategorien eine Rangfolge oder einen numerischen Wert zuzuweisen. (Geschlecht, Haarfarbe, etc)
    \item \textbf{ordinal}: Die Ausprägungen haben eine Rangordnung, allerdings ist der Abstand zwischen den Ausprägungen nicht gleichmäßig oder gar unbekannt. (Schulnoten (1, 2, 3, 4, 5, 6), Zufriedenheitsstufen (sehr zufrieden, zufrieden, unzufrieden, sehr unzufrieden) oder sozioökonomischer Status (niedrig, mittel, hoch))
\end{itemize}

}

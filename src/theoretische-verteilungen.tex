\chapter{Theoretische Verteilungen}
Theoretische Verteilungen bilden den Übergang von der deskriptiven zur induktiven Statistik und sind eng mit den Prinzipien der Wahrscheinlichkeitsrechnung verbunden.
Im Gegensatz zu empirischen Verteilungen, die auf der Beobachtung und Zusammenfassung von tatsächlich erhobenen Daten basieren, sind theoretische Verteilungen mathematische Modelle, die die Wahrscheinlichkeiten von Ergebnissen beschreiben, basierend auf bestimmten Annahmen oder Theorien.
\newline \newline
In der Praxis stellen theoretische Verteilungen idealisierte oder vereinfachte Darstellungen der Realität dar und dienen als nützliche Werkzeuge für die Durchführung von statistischen Tests und Schätzungen.
Sie ermöglichen es uns, Hypothesen zu prüfen und Schlussfolgerungen über eine Population basierend auf einer Stichprobe zu ziehen.
Einige der bekanntesten theoretischen Verteilungen sind die Normalverteilung, die Binomialverteilung und die Poisson-Verteilung, die jeweils verschiedene Arten von Daten und Situationen modellieren.
\newline \newline
Bei den theoretischen Verteilungen heißen die Merkmalsausprägungen \textbf{Ereignisse} und die relativen Häufigkeiten \textbf{Wahrscheinlichkeiten.}


\section{Zufallsvariablen}

Zufallsvariablen sind ein grundlegendes Konzept in der Wahrscheinlichkeitstheorie und Statistik. Formal betrachtet, ist eine Zufallsvariable eine messbare Funktion $X$, die jedem Ergebnis $\omega$ aus einem Wahrscheinlichkeitsraum $(\Omega, \mathcal{F}, P)$ eine reelle Zahl zuordnet. Das bedeutet, dass die Zufallsvariable $X$ den Ergebnissen aus dem Ereignisraum $\Omega$ bestimmte Werte zuweist, die wir messen oder beobachten können.

\textit{Zufallsvariablen können in zwei Haupttypen unterteilt werden: diskrete und stetige Zufallsvariablen.}

\textbf{Diskrete Zufallsvariablen} haben eine Zählmenge von möglichen Ergebnissen. Beispiele hierfür sind Würfelwürfe, die Anzahl der Münzwürfe bis zum ersten Kopf oder die Anzahl der Personen, die an einem bestimmten Tag in einem Geschäft einkaufen.

\textbf{Stetige Zufallsvariablen} hingegen können jeden Wert innerhalb eines bestimmten Bereichs annehmen, wie zum Beispiel die Zeit, die eine Person auf einen Bus wartet, oder das Gewicht einer zufällig ausgewählten Person.

\textit{Ein grundlegendes Beispiel für eine diskrete Zufallsvariable ist das Werfen eines fairen Würfels.}

In diesem Fall ist der Wahrscheinlichkeitsraum gegeben durch $\Omega = \{1, 2, 3, 4, 5, 6\}$, das sind die möglichen Ergebnisse (Augenzahlen) beim Würfeln. Wir definieren die Zufallsvariable $X$ als die Augenzahl, die beim Würfeln auftritt. Die Wahrscheinlichkeit für jedes Ergebnis ist gleich und beträgt $\frac{1}{6}$, da wir einen fairen Würfel verwenden.

Die Verteilung dieser Zufallsvariablen kann in der folgenden Tabelle dargestellt werden:

\begin{center}
    \begin{tabular}{c|c}
        $x_i$ & $P(X = x_i)$  \\
        \hline
        1     & $\frac{1}{6}$ \\
        2     & $\frac{1}{6}$ \\
        3     & $\frac{1}{6}$ \\
        4     & $\frac{1}{6}$ \\
        5     & $\frac{1}{6}$ \\
        6     & $\frac{1}{6}$ \\
    \end{tabular}
\end{center}

Diese Tabelle zeigt uns die Wahrscheinlichkeit dafür, dass die Zufallsvariable $X$ einen bestimmten Wert annimmt. Es ist wichtig zu verstehen, dass diese Wahrscheinlichkeiten theoretische Wahrscheinlichkeiten sind, die auf der Annahme basieren, dass der Würfel fair ist.
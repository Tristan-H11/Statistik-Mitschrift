\chapter{Einführung}
\section{Geschichtliches}
\ex{}{
"Wenn ein Mensch stirbt, ist es ein Malheur, bei 100 Toten ist es eine Katastrophe, ab 1000 Toten eine Statistik." - Eichmann \newline
Diese Form von Statistiken als Erbsenzählerei gibt es schon sehr lange. Die ersten geschrieben Texte verweisen auf Zahlen, Statistiken wurden des Pyramidenbaus angefertigt,
die Bibel beschreibt eine Volkszählung. \newline
Man kann sich nur schwer eine Zeit ohen Zahlen vorstellen, es muss sie aber gegeben haben. Dann haben die Leute Sachen eben nicht durchnummeriert, sondern wahrscheinlich Namen gegeben.
Statt "es fehlen 2 Tiere" haben sie vermutlich "es fehlen Peter und Paul" gesagt.
}
Es gibt aber auch eine Definition von Statistik im instrumentalen Sinne: Verfahren, nach denen empirische Zahlen gewonnen, dargestellt, verarbeitet, analysiert und für Schlussfolgerungen, Prognosen und Entscheidungen verwendet werden.
\newline \newline
Diese beiden Definitionen sind nacheinander entstanden. Erst hat man Jahrhunderte lang Daten gesammelt, bis irgendwann bis Notwendigkeit kam, Verfahren zu entwickeln, um die Übersicht zu erhöhen, ohne allzu viel Information zu verlieren.

\clm{Relevanz von Daten}{}{Ohne Daten machen statistische Analyse und deren Methoden keinen Sinn.}

Es gibt drei verschiedene Formen des Wissens oder der Erkenntnisgewinnung: \textit{Wissen durch Wahrnehmen, Wissen durch Logik oder Schlussfolgern und Wissen duch Glauben. Diese Formen des Wissens wirken zusammen und beeinflussen sich wechselseitig beim Erzeugen und Verwenden von Wissen.} Leider ist \textit{Wissen durch Glauben vielleicht die häufigste Form des Wissens.}
\newline
Die statistische Analyse selber hat mit den beiden ersten Formen des Wissens zu tun. Sie gibt die Instrumente, Informationen zu überprüfen, um sie nicht einfach glauben zu müssen.

\section{Erste Anwendungen}
\dfn{Deskriptive und empirische Statistik}{Die Analyse der (Daten-)Vorgänge oder Verteilungen innerhalb der Grundgesamtheit bzw innerhalb der Stichprobe nennt man \textbf{deskriptive} oder \textbf{beschreibende} Statistik. \newline
Wird die Verteilung nicht analysiert, sondern \textbf{erhoben}, so wird sie \textbf{empirisch} genannt.
}
Für die Grundgesamtheit ist eine ständige Erhebung oft nicht machbar. Deshalb wird versucht, die Verteilung zu modellieren. Also eine Funktion zu definieren, welche die Verteilung möglichst gut beschreibt. Man spricht dann von einer \textbf{theoretischen} Verteilung. Ein berühmtes Beispiel wäre die Normalverteilung.
\newline \newline
Ist die Grundgesamtheit bekannt und will man im voraus Aussagen über die zu ziehende Stichprobe treffen, spricht man von \textbf{deduktiver Statistik}. Dies ist im bekanntesten Fall die Wahrscheinlichkeitsrechnung. \newline
Ist umgekehrt nur die Stichprobe bekannt und will man eine Aussage über die unbekannte Grundgesamtheit treffen, spricht man von \textbf{Inferenzstatistik oder deduktiver Statistik}. Weitere Synonyme für diese sind \textit{analytische Statistik} oder \textit{beurteilende Statistik}.
\ex{Beispielhafte Zuordnung von Inferenzstatistik und deskriptiver Statistik}{
\begin{itemize}
    \item \textit{Mitteilung über die Zahl der Schüler, die in einer Klassenarbeit gut oder schlecht abgeschnitten haben}: Deskriptiv.
    \item \textit{Berechnung eines Stichprobenmittelwertes zur Bezeichnung der zentralen Tendenz in den Daten}: Deskriptiv.
    \item \textit{Durchführung einer Untersuchung, um den Zusammenhang zwischen Bildungsniveau und Einkommen in der BRD zu bestimmen}: Inferenzstatistik.
    \item \textit{Angabe zum Durchschnittsgehalt von Beamten in den neuen Bundesländern auf der Basis der gesamten Gehaltsstatistik}: Deskriptiv.
    \item \textit{Schätzung der durchschnittlichen Regenmenge in 1999 in Hannover}: Inferenzstatistik
\end{itemize}
}
